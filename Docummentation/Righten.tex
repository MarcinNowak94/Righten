% Docummentation:
% - https://www.latex-project.org/help/documentation/
% - https://docs.w3cub.com/latex/
% - https://tex.stackexchange.com/questions/455993/formatting-sql-code
% - https://www.overleaf.com/learn/latex/Code_listing#Code_styles_and_colours

%--------------------------------------------------------------------------------
%-----------------------------------   TODO   ----------------------------------
%--------------------------------------------------------------------------------
% add section about icon source https://www.iconpacks.net/free-icon/money-bag-6384.html


\documentclass[a4paper,10pt, twoside]{report}

\usepackage{polski}         % Polish diacretic signs
\usepackage[utf8]{inputenc} % required for international characters
\usepackage{hyperref}       % urls and hyperlinks
\usepackage{xurl}           % break urls
\usepackage{microtype}      % improve justification
\usepackage{enumitem}       % compact lists
\usepackage{graphicx}       % Include graphics
\usepackage{wrapfig}        % wrap text aroung graphics
\usepackage{fancyhdr}       % Customize page layout
\usepackage{index}          % Create an index
\usepackage{setspace}       % Spacing
\usepackage{float}          % Forcing figure placement
\usepackage{placeins}       % Barrier to floats so tables don't "sink
\usepackage{tabularray}     % Tables with wrapping
\usepackage{xcolor,listings}% Code listings
\usepackage{textcomp}       % Code listings
\usepackage{color}          % Code listings
\usepackage{nameref}        % Reference chapter, section, etc. by name
\usepackage{afterpage}
\usepackage{tcolorbox}      % Colored boxes for inline Code

\makeindex
\graphicspath{ {./} }       %Was ./figures/ changed so i can peek in VS Code
%Widow and orphan control as per https://tex.stackexchange.com/questions/4152/how-do-i-prevent-widow-orphan-lines/4167#4167?newreg=87e476a8b11e44e2bd285df0190a69e0
\widowpenalty10000
\clubpenalty10000

% Code listings
\definecolor{codegreen}{rgb}{0,0.6,0}
\definecolor{codegray}{rgb}{0.5,0.5,0.5}
\definecolor{codepurple}{HTML}{C42043}
\definecolor{backcolour}{HTML}{F2F2F2}
\definecolor{bookColor}{cmyk}{0,0,0,0.90}
\color{bookColor}

\lstset{upquote=true}

\lstdefinestyle{mystyle}{
    backgroundcolor=\color{backcolour},
    commentstyle=\color{codegreen},
    keywordstyle=\color{codepurple},
    numberstyle=\numberstyle,
    stringstyle=\color{codepurple},
    basicstyle=\footnotesize\ttfamily,
    breakatwhitespace=false,
    breaklines=true,
    captionpos=b,
    keepspaces=true,
    numbers=left,
    numbersep=10pt,
    showspaces=false,
    showstringspaces=false,
    showtabs=false,
}
\lstset{style=mystyle}

% Code listings
\newcommand\numberstyle[1]{
    \footnotesize
    \color{codegray}
    \ttfamily
    \ifnum#1<10 0\fi#1 |
}

\renewcommand{\lstlistlistingname}{Listingi}

% ------------------------------ Custom Commands ------------------------------
% Usage: \command\{text}  
\newcommand{\customstyletitle}[1]{\Huge{\textbf{#1}}}
\newcommand{\customstylechapter}[1]{\large{\textit{#1}}}
\newcommand{\customstylesection}[1]{\textbf{\textit{#1}}}
\newcommand{\customstylesidenote}[1]{\Small{\textbf{#1}}}
\newcommand{\customstyletable}[1]{\footnotesize{\textbf{#1}}}
\newcommand{\customstyletablecentered}[1]{\footnotesize\centering{\textbf{#1}}}
\newcommand{\customstyleindivisible}[1]{
    \begin{minipage}{\textwidth}
        {#1}
    \end{minipage}
}

\lstnewenvironment{SQLlisting}[2][]%
  {\noindent\minipage{\linewidth}\medskip
   {#2}
   \smallskip
   \lstset{basicstyle=\ttfamily\footnotesize,
            frame=single,
            language=SQL,
            deletekeywords={IDENTITY},
            %deletekeywords={[2]INT},
            morekeywords={clustered},
            framesep=8pt,
            xleftmargin=40pt,
            framexleftmargin=40pt,
            frame=tb,
            framerule=0pt,
            caption=#1}}
  {\endminipage}

  % spacer
\newcommand{\HRule}{\rule{\linewidth}{0.5mm}} % horizontal lines

\newtcbox{\inlinecode}{
    on line,
    boxrule=0pt,
    boxsep=0pt,
    top=2pt,
    left=2pt,
    bottom=2pt,
    right=2pt,
    colback=gray!15,
    colframe=white,
    fontupper={\ttfamily \footnotesize}
}

% --------------------------- documment starts here ---------------------------

% environment
\begin{document}
\begin{large}       %Remark 1: make size bigger by ~1,5p

% Define pagestyle
% [REQUIREMENT] 23. Przypisy dolne, stopki (nr stron), nagłówki...
\pagestyle{fancy}
\fancyhf{}          %clears default headers and footers
\renewcommand{\headrulewidth}{2pt}
\renewcommand{\footrulewidth}{1pt}

\fancypagestyle{mychapterpage}{%
    %\fancyhead[LE]{\leftmark}
    %\fancyhead[RO]{\rightmark}
    \fancyfoot[RO,LE]{\thepage}
    \renewcommand{\headrulewidth}{2pt}
    \renewcommand{\footrulewidth}{1pt}
}


% https://texblog.org/2013/09/16/multiple-page-styles-with-fancyhdr/
%Redefine chapter by adding fancy as the chapter title page page-style
\makeatletter
    \let\stdchapter\chapter
    \renewcommand*\chapter{%
    \@ifstar{\starchapter}{\@dblarg\nostarchapter}}
    \newcommand*\starchapter[1]{%
        \stdchapter*{#1}
        \thispagestyle{mychapterpage}
        \fancyfoot[RO,LE]{\thepage}
        \markboth{\MakeUppercase{#1}}{}
    }
    \def\nostarchapter[#1]#2{%
        \stdchapter[{#1}]{#2}
        \thispagestyle{mychapterpage}
        \fancyfoot[RO,LE]{\thepage}
    }
\makeatother


% [REQUIREMENT] 1. Strona tytułowa
\begin{titlepage}
	%---Headings------------------------------------------	
	\begin{center}
    \begin{onehalfspace}
    \textsc{\LARGE{WYŻSZA SZKOŁA TECHNOLOGII INFORMATYCZNYCH W KATOWICACH}}\\
    \end{onehalfspace}
    \textsc{\large{WYDZIAŁ INFORMATYKI}}\\
	\textsc{\large{KIERUNEK: INFORMATYKA}}\\
    \end{center}
    
    %---Author--------------------------------------------
	\begin{flushleft}
    \textsc{Nowak Marcin}\\[0cm]
    \textsc{Nr Albumu 08255}\\[0cm]
    \textsc{Studia niestacjonarne}\\[0cm]
    \end{flushleft}
	
    %---Title---------------------------------------------
	\begin{center}
    \HRule\\[0.4cm]
	{\customstyletitle{Projekt i implementacja aplikacji wspomagającej zarządzanie budżetem domowym}}\\[0.4cm] 
    \HRule\\[1.5cm]
    \end{center}
	
    %---Description----------------------------------------
	\begin{flushright}
        \textsc{Przedmiot: Projekt Systemu Informatycznego}\\[0cm]
        \textsc{pod kierunkiem}\\[0cm]
        \textsc{mgr. Jacek Żywczok}\\[0cm]
        \textsc{W roku akademickim 2022/23}\\[0cm]
    \end{flushright}
 
	%---Date & logo---------------------------------------
	\vfill                  % Position the date lower
	\begin{center}
    {Katowice 2022}\\	    % \today
	\includegraphics[width=0.2\textwidth]{figures/WSTI-logo.jpg}\\[1cm]
	\end{center}
\end{titlepage}

\null\newpage % #TODO: fix, this moves formatting \addtocounter{page}{-1}

% [REQUIREMENT] 2. Spis treści
\renewcommand*\contentsname{Spis treści}
\tableofcontents                    % prints automatical table of contents

% ---------------------------------- Content ----------------------------------


% 3.1. Wstęp <-[Wprowadzenie do tematyki projektu, Zamierzony cel projektu]
%http://siminskionline.pl/seminarium-inzynierskie/struktura-pracy-inzynierskiej/wstep/
\chapter{\customstylechapter{Wstęp}}
{Finanse są dziedziną nauki ekonomicznej która zajmuje się rozporządzaniem 
pieniędzmi \cite{wiki_ekonomia}. Nauka ta w podobnym zakresie a różnej skali 
dotyczy państw, dużych przedsiębiorstw, małych działalności gospodarczych jak i 
zwykłych obywateli - w efekcie jest to dziedzina o stosunkowo prostych 
podstawach jednak niesamowicie skomplikowana w każdym aspekcie w którym można ją
 zagłębić. Wiedza z tego zakresu staje się szczególnie przydatna w momencie 
dynamicznych zmian sytuacji ekonomicznej, wtedy nierzadko decyduje ona o jakości
 oraz stanie życia poszczególnych osób fizycznych, rentowności przedsiębiorstw 
czy stabilności państw \cite{zapaśćekonomiczna}. W przypadku państw i firm 
przeważnie do zarządznia budżetem oddelegowane są dedykowane całe zespoły lub 
dedykowani eksperci z tej dziedziny. Jednak osoby zarządzające budżetem domowym 
najczęściej dysponują wyłacznie nabytym doświadczeniem i na ogół stosują 
podejście intuicyjne, rzadko jeśli wogóle wspomagając się jakimikolwiek 
narzędziami które ułatwiałyby to zadanie. Część z nich może poszukiwać 
pożytecznych treści o tematyce finansowej w Internecie, jednakże rozpoczynając 
zaznajamianie się z tematyką mogą mieć spore trudność ich przystępnością oraz 
wyłuskaniem źródeł dobrej jakości informacji w natłoku materiałów błędnych, 
słabych merytorycznie, niekatualnych czy też nastawionych na marketing ponad 
poprawność.}

\medskip
{Celem pracy jest zaprojektowanie i realizacja modułu analitycznego aplikacji 
która ułatwi jej użytkownikom zarządzanie budżetem domowym poprzez dostarczenie 
narzędzia do analizy wpływów i wydatków, wizualizacji trendów oraz automatycznie 
kategoryzujące wprowadzone dane. W zamierzeniu aby ułatwić obsługę wymagać 
będzie minimalnej wiedzy i konfiguracji ze strony użytkownika, dostarczając mu 
jednocześnie możliwie najlepsze narzędzia. Będzie to aplikacja przeglądarkowa 
napisana w języku Python \cite{Python}\cite{pythonautomate}, wykorzystująca 
frameworki: Bootstrap \cite{Bootstrap}, Flask \cite{Flask}, WTForms 
\cite{WTForms}, jinja2 \cite{jinja}, oraz bibliotekę chart.js \cite{chart.js}.}

\medskip
{Rozdział drugi zawiera rozwinięcie charakterystyki i motywacji problemu krótko 
zaznaczonej we wstępie pracy. Rozdział trzeci to analiza istniejących rozwiązań.
 Rozdział czwarty opisuje koncepcję własnego rozwiązania. Rozdział piąty to 
ogólny zarys projektu. W rozdziale szóstym znajduje się dokumentacja techniczna.
Rozdział siódmy to opis testów i weryfikacji systemu. Rozdział ósmy opisuje 
przykładowy scenariusz wykorzystania systemu. Ostatecznie rozdział dziewiąty 
jest zakończeniem pracy.}

% 3.2. Charakterystyka/analiza problemu
%http://siminskionline.pl/seminarium-inzynierskie/struktura-pracy-inzynierskiej/charakterystykaanaliza-problemu/
\chapter{\customstylechapter{Charakterystyka i analiza problemu}}

{Na dzień dzisiejszy wiedza z zakresu finansów oferowana w ramach systemu 
edukacji publicznej jest znikoma \cite{edukacjafinansowawszkołach}. Sytuacja ta 
utrzymuje się od dawna, dlatego spora część obywateli Polski słabo orientuje się
 w kwestii finansów osobistych, ekonomii i przedsiębiorczości. Istnieje wiele 
aktywnie działających programów które wprowadzają uczestników w świat finansów 
poprzez przedstawienie podstawowych zagadnień z dziedziny ekonomii i podstaw 
inwestowania \cite{edukacjafinansowawszkołach}. Część działań ma na celu zbudować 
w uczestnikach świadomość ogólnej sytuacji ekonomicznej jednakże jak wynika z 
badań Banku Pekao \cite{edukacjafinansowamlodziezy}, większość rodziców stwierdza
 że nie posiada wystarczającej wiedzy o finansach żeby przekazać ją dzieciom, 
co pozwala wysnuć wniosek iż sami zarządzają finansami rodzinnymi korzystając 
raczej z intuicji i własnego doświadczenia aniżeli solidnych podstaw 
teoretycznych. Tego rodzaju podejście na wyczucie, działa przez większość czasu, 
wydaje się że nie ma większego wpływu na życie gdy sytuacja ekonomiczna jest 
spokojna - zmiana podejścia pozwala wtedy co prawda więcej zaoszczędzić, jednak 
w zasadzie jest to opcjonalne. Kiedy jednak na rynku czy to lokalnym czy 
globalnym sytuacja staje się trudniejsza, co może powodować nagły wzrost 
inflacji przeważnie koszta życia rosną niewspółmiernie do 
zarobków \cite{gussytuacjabudzetowa}\cite{koszty2010-22}, a tym samym dopięcie 
finansów osobistych i domowych tak, by bilans wyszedł dodatni wymaga więcej 
uwagi i wiedzy. W ostatnich latach (2019-2023) miało miejsce kilka zdarzeń któe 
dotknęły światową gospodarkę. W momencie pisania tej pracy takimi wydarzeniami 
są pandemia Covid 19, działania zbrojne nw terenie Ukrainy oraz wojna na bliskim
 wschodzie. Zdarzenia te jak wynika z badań Krajowego Rejestru Długów 
\cite{portfelpolakawpandemii} u prawie połowy polaków wywołała poczucie 
zagrożenia biedą, podczas gdy jedynie 22\% twierdzi że jest spokojna o swoją 
sytuację finansową. Raport Warsaw Enterprise Institute \cite{weiinflacja} 
wykazuje natomiast spadek realnych płac (uwzględniających zarobki oraz wydatki) 
średnio o 2\%, a w niektórych grupach możliwe 5-11\% w latach 2020-2022 co 
wskazuje na wysokie zapotrzebowanie na narzędzia i edukację w zakresie 
budżetowania.}

\begin{figure}[H]           %requires float package
    \centering
    \includegraphics[width=12cm]{figures/GUS_dynamikarealnychdochodowiwydatkow2010-2020.png}
    \caption{Główny Urząd Statystyczny, Dynamika realnych dochodów i wydatków 
    na 1 osobę w gospodarstwach domowych w latach 2010–2022 \cite{gussytuacjabudzetowa}}
    \label{fig:gusbudzet10-22}
\end{figure}

\begin{figure}[H]           %requires float package
    \centering
    \includegraphics[width=12cm]{figures/300gospodarka-pl_inflacjawpolsce.png}
    \caption{300gospodarka.pl, Inflacja w Polsce w latach 1996-2023 dane GUS}
    \label{fig:inflacja}
\end{figure}

{Planowanie domowego budżetu jest podstawowym narzędziem które pomaga utrzymać 
wydatki w ryzach. Składa się z dwóch etapów - po pierwsze metody lub narzędzia 
które ułatwią zarządzanie budżetem domowym oraz znajomości podstaw zarządzania 
finansami. Dlatego dla osób rozpoczynających budżetowanie ważne jest 
przedstawienie w możliwie prostej, zwięzłej i przystępnej formie już gotowych 
opracowanych rozwiązań które można zastosować aby świadomie zarządzać sytuacją 
finansową własnego domostwa. W najprostszym wariancie na 
budżet \cite{o24_budzetowanie}\cite{budget}\cite{iwućbudżet}\cite{mintbudget}
\cite{ingbudżet}\cite{budzetdomowypodkontrola} składają się: wpływy czyli 
dochody ze wszystkich źródeł, zobowiązania czyli płatności stałe jak rachunki 
czy raty kredytów oraz wydatki które są zróżnicowane. Skrupulatne zbieranie 
danych z pewnego okresu pozwala określić ogólną sytuację, a w miarę wydłużania 
zakresu czasu dostępnych danych i zwiększania ich precyzja wyłaniają się trendy 
co umożliwia prognozowanie przyszłej sytuacji. Istnieje wiele różnych podejść do
 tworzenia budżetu - od najbardziej ogólnych które skupiają się wyłącznie na 
określeniu bilansu wydatków oraz wpływów, po najbardziej szczegółowe analizy 
wydatków na poszczególne kategorie czy nawet produkty. Każde z podejść ma swoje 
dobre strony, i w gruncie rzeczy wybór odpowiedniego podejścia jest wyłącznie 
kwestią preferencji.}

\begin{figure}[H]           %requires float package
    \centering
    \includegraphics[width=12cm]{figures/oszczedzaniepieniedzyblog-pl_wydatki.jpg}
    \caption{oszczedzaniepieniedzyblog.pl, Najprostszy budżet - wydatki}
    \label{fig:prostybudżetwydatki}
\end{figure}

\begin{figure}[H]           %requires float package
    \centering
    \includegraphics[width=12cm]{figures/oszczedzaniepieniedzyblog-pl_przychody.jpg}
    \caption{oszczedzaniepieniedzyblog.pl, Najprostszy budżet - przychody}
    \label{fig:prostybudżetprzychody}
\end{figure}

{Poza ukazaniem ogólnego obrazu sytuacji finansowej w budżecie uwzględnić można 
cele jak spłata zadłużenia, czy planowany znaczny wydatek, oraz limity 
pomagające ograniczyć wydatki i cele przychodów zwiększające ilość dostępnych 
środków finansowych. W literaturze przedmiotowej opisano także wiele przydatnych
 podejść oraz zasad jak wstępny podział wydatków na podstawie 
priorytetów \cite{najbogatszyczlowiekwbabilonie}\cite{budzetdomowypodkontrola} 
czy uwzględnienie oszczędzania w formie podejścia najpierw zapłać sobie 
\cite{najbogatszyczlowiekwbabilonie}\cite{finansowaforteca} które można 
zastosować jako strategie zarządzania finansami domowymi aby usprawnić budżet 
lub osiągnąć zamierzony cel. Jednak aby zastosować daną strategię trzeba ją 
najpierw znać, a jak wynika z informacji opisanych wcześniej poziom wiedzy z 
zakresu finansów w Polsce oceniany jest jako słaby, dlatego narzędzia do 
zarządzania finansami powinny, w najprostszej formie udostępniać takie 
informacje w łatwo przystępnej formie, lub przy bardziej zaawansowanym podejściu
 posiadać wbudowane mechanizmy które pozwoliłyby użytkownikowi wybrać i 
zastosować strategię bez potrzeby jej dogłębnej znajomości.}

{Z uwagi na tematykę zwyczajowo są to dane bardzo wrażliwe, zatem wymagają 
odpowiednich zabezpieczeń. Idealną opcją dla potencjalnych użytkowników byłoby 
gdyby jako jedyni mieli dostęp do prywatnych danych, oraz mogli sami precyzyjnie
 decydować komu je udostępniają. Przy znaczącej statystycznie liczbie 
użytkowników dane zebrane w aplikacji po odpowiedniej pełnej nieodwracalnej 
anonimizacji i uśrednieniu mogą posłużyć do modelowania wydatków obywateli 
danych regionów, sektorów, segmentów gospodarki lub nawet ogólnie całego 
państwa, co z kolei można wykorzystać zwrotnie w samej aplikacji aby porównać 
model wydatków użytkownika do adekwatnej średniej i zwrócić uwagę na obszary w 
których pozytywnie od niej odstaje działając jako pozytywne wzmocnienie 
dobrego nawyku \cite{pozytywnewzmocnienie}. Tego typu modelowanie jest już 
przeprowadzane przez Główny Urząd Statystyczny, którego raporty publikowane są 
co roku, jest to więc potencjalne źródło najbardziej precyzyjnych danych, które 
mogłoby zostać wykorzystane do porównań.}


% 3.3. Analiza istniejących rozwiązań
% http://siminskionline.pl/seminarium-inzynierskie/struktura-pracy-inzynierskiej/analiza-istniejacych-rozwiazan/
% Pull from documment: https://docs.google.com/document/d/1cYnRqie6xpVnRh8HOuYQZlcuSP_wxj1rPUvHp9FGgDg/edit
\chapter{\customstylechapter{Analiza istniejących rozwiązań}}
{W tym rozdziale przedstawiona zostanie analiza obecnie dostępnych rozwiązań 
opisanego problemu aby określić ich wady i zalety. Jako że budżetowanie jest 
problemem tak starym jak sam wynalazek pieniądza, historycznie powstało wiele 
różnych rozwiązań których celem jest je ułatwić.}

{Podstawową i najprostszą formą budżetu jest zapis na papierze czy chociażby w 
formie księgi zawierającej przychody i wydatki \cite{o24_budzetowanie}. Sposób ten zostanie 
przeanalizowany ponieważ do niedawna była to główna metoda prowadzenia budżetu 
i mimo postępu cyfryzacji i informatyzacji nadal jest szeroko stosowany. Po 
części zastąpiły go rozwiązania komputerowe w formie różnych aplikacji które 
wymagają mniejszych lub większych nakładów pracy od użytkownika. Okazjonalnie 
tego typu zestawienia prowadzone są dziś także w arkuszach kalkulacyjnych.}

{Pierwszą w pełni cyfrową opcją są same witryny kont bankowych \cite{ingbudżet} 
na których klient często może kategoryzować poszczególne transakcje i wyświetlać
 podsumowania oraz określić zakładany budżet. Rozwiązania te są dostępne dla 
każdego klienta danego banku dlatego warto się im przyjrzeć, jako przykład 
posłuży portal banku Santander - centrum24.pl \cite{santandercentrum24}.}

{Na kolejną kategorię rozwiązań składają się aplikacje dedykowane do zarządzania
 budżetem \cite{budget}. Systemy tego typu po wprowadzeniu danych udostępniają 
użytkownikowi cały wachlarz dodatkowych specjalistycznych opcji i narzędzi. 
Omówione zostaną dwa przykłady tego rodzaju aplikacji - Intuit mint \cite{mint} 
oraz Goodbudget \cite{goodbudget}. Na rynku dostępnych jest wiele więcej 
rozwiązań przez co użytkownik ma dowolny wybór, jednak świadomy wybór 
odpowiedniej opcji wymaga od użytkownika dokładnego przeglądu i porównania kilku
 aplikacji.}

%#TODO: Describe what criteria will be evaluated
\section{\customstylesection{Budżet papierowy lub arkusz kalkulacyjny}}
{Grupa ta obejmuje wiele różnorodnych narzędzi, nierzadko darmowych, lub takich, 
które użytkownik posiada do innych celów. Przykładowe opcje obejmują proste 
rozpiski i podsumowania na kartkach, arkusze kalkulacyjne jak Microsoft 
Excel, Google Sheets, LibreOffice Calc - przykłady na rysunkach 
\ref{fig:budzetprzykladowyexcel} oraz \ref{fig:budzetprzykladowypapier}. Największą 
zaletą tych rozwiązań jest prostota dzięki której z tego typu rozwiązania jest 
w stanie skorzystać w zasadzie każdy jednakże odpowiedzialność za manualne 
utrzymanie i dbanie o jakość czy spójność danych spoczywa wyłącznie na 
użytkowniku który jest jednocześnie autorem budżetu. Jedynie w podejściach 
cyfrowych okazjonalnie znaleźć można dodatki służące do automatyzacji części 
funkcji, niemniej jednak użytkownicy posiadający odpowiednią wiedzę i 
umiejętności mogą przygotować tego typu mechanizmy osobiście - przykładowo w 
arkuszach kalkulacyjnych jako formuły czy nawet skrypty (np. VBA w Excell) 
jeżeli narzędzie ma taką funkcję. Udostępniają także każde możliwe podejście 
znane użytkownikowi oraz całkowitą wolność wyboru chociażby kategoryzacji. Z 
uwagi na niski próg wejścia w sieci dostępnych jest wiele poradników oraz 
szablonów, choć paradoksalnie jednocześnie jest to wada ponieważ początkującemu 
użytkownikowi trudno się odnaleźć w sporej ilości prezentowanych opcji. Zależnie
 od wykorzystanej technologii (uwzględniając także budżet papierowy) mogą być 
dostępne zarówno lokalnie jak i w przeglądarce.}

{Na uwagę zasługuje również fakt iż podejście takie niejako mimochodem uczy 
użytkownika podstaw finansów i zmusza do refleksji nad swoją sytuacją, co może 
zaowocować wypracowaniem własnych spersonalizowanych systemów dostosowanych pod 
swoje potrzeby i dopasowanych do preferowanych metod pracy lepiej niż pozostałe 
dostępne rozwiązania.}

\begin{figure}[H]           %requires float package
    \centering
    \includegraphics[width=12cm]{figures/marciniwuc.com_Wydatki-nieregularne-plan-roczny.png}
    \caption{https://marciniwuc.com/budzet-domowy-05-wydatki-nieregularne/ Przykładowy budżet w aplikacji Excel}
    \label{fig:budzetprzykladowyexcel}
\end{figure}

{Co ciekawe na tym podejściu zbudowana jest także aplikacja Tiller \cite{tiller},
 służy jako interfejs który pozwala użytkownikowi na migrację danych bankowych 
do prywatnego arkusza kalkulacyjnego w Excel lub Google Sheets, w którym tworzy 
predefiniowany szablon z wizualizacjami i podsumowaniami.}

\begin{figure}[H]           %requires float package
    \centering
    \includegraphics[width=12cm]{figures/jakoszczedzacpieniadze.pl_budzet-domowy-1.jpg}
    \caption{https://jakoszczedzacpieniadze.pl/prosty-budzet-domowy Przykładowy budżet na papierze}
    \label{fig:budzetprzykladowypapier}
\end{figure}

\section{\customstylesection{Systemy bankowe - Przegląd wydatków w Santander}}
{Wraz z rozwojem technologii i cyfryzacji posiadanie konta bankowego stało się 
praktyczne wymogiem. Na rynku istnieje szeroki wybór, dlatego jak każda 
nowoczesna firma, także i banki starają się wyjść naprzeciw oczekiwaniom 
klienta i zachęcić go dodatkowymi funkcjami. W efekcie, choć jest to jedynie 
opcja dodatkowa, część z nich wdrożyło u siebie usługę pomagającą użytkownikowi 
w analizie finansów i planowaniu budżetu.}

\begin{figure}[H]           %requires float package
    \centering
    \includegraphics[width=12cm]{figures/Santander_PrzegladWydatkow_historia.png}
    \caption{Centrum24.pl, Santander historia transakcji}
    \label{fig:santanderhistoria}
\end{figure}

{Jako przykład posłuży system Przegląd wydatków na portalu centrum24.pl banku 
Santander zaprezentowany na rysunku \ref{fig:santanderprzeglad}. Usługa stara się 
na bieżąco przypisywać transakcje dokonane na koncie klienta do dość ogólnych 
kategorii, jednocześnie użytkownik może je dowolnie zmienić. Kategorie mają dwa 
poziomy szczegółowości, dzięki czemu są grupowane jednak w pewnym stopniu 
pozwalają ukazać szczegóły. Jak widać na rysunku \ref{fig:santanderkategoryzacja} 
użytkownikowi prezentowany jest wykres słupkowy pokazujący ilość wydanych 
pieniędzy na poszczególne kategorie lub grupy kategorii, oraz wykres pokazujący 
sumę wydatków w danym okresie i wszystkie transakcje które wchodzą w ich skład.}

\begin{figure}[H]           %requires float package
    \centering
    \includegraphics[width=12cm]{figures/Santander_PrzegladWydatkow_przeglad_kategoria.png}
    \caption{Centrum24.pl, Santander przegląd kategorii wydatków}
    \label{fig:santanderprzegladkategoria}

    \includegraphics[width=12cm]{figures/Santander_PrzegladWydatkow_kategoryzacja.png}
    \caption{Centrum24.pl, Santander kategoryzacja wydatków}
    \label{fig:santanderkategoryzacja}
\end{figure}

{System ma także swoje ograniczenia. Niestety prezentuje wyłącznie wydatki, tym 
samym jedyne miejsce w którym użytkownik może podejrzeć swoje przychody jest 
historia konta, jednak jak widać na rysunku \ref{fig:santanderhistoria} jest to 
jedynie suma w wybranym okresie, ewentualnie zestawienie transakcji wpływów 
pozostawiając analizę w gestii użytkownika.}

\begin{figure}[H]           %requires float package
    \centering
    \includegraphics[width=12cm]{figures/Santander_PrzegladWydatkow_przeglad.png}
    \caption{Centrum24.pl, Santander przegląd wydatków}
    \label{fig:santanderprzeglad}
\end{figure}

\section{\customstylesection{Aplikacje dedykowane - Intuit Mint}}
{W trakcie pisania pracy dostawca aplikacji zdecydował się wycofać aplikacje z 
początkiem roku 2024 i zachęcić użytkowników do migracji na swoją platformę 
Credit Karma \cite{mintwycofanie} która pozbawiona jest funkcji budżetowania. 
Mimo to przykład pozostaje aktualny ponieważ aplikacja jest uznawana przez wielu
 użytkowników i recenzentów za jedną z najlepszych w kategorii finansów, 
dlatego warto zwrócić uwagę na jej zalety i wady, zwłaszcza że jej wycofanie 
tworzy na rynku pewną niszę.}


{Aplikacja Intuit Mint \cite{mint}\cite{mintrecenzja} dostępna jest w wersji 
darmowej z reklamami lub płatnej, na systemy mobilne Android i iOS oraz 
przeglądarki. Po uruchomieniu Mint prezentuje użytkownikowi ekran z 
podsumowaniem finansów w obecnym miesiącu w postaci zakładek które grupują 
informacje z kilku kategorii, m.in.: Wartość netto, Wydatki, Inwestycje. Aby 
zasilić dane użytkownik musi dać aplikacji dostęp do swoich kont bankowych i 
inwestycyjnych, co wiele recenzentów uznaje za niesamowicie wygodne jednak 
jednocześnie przez to nie jest to aplikacja dla ludzi dbających o prywatność - 
jest to także główny powód dlaczego przegląd tej aplikacji oparty jest na 
recenzjach i poradnikach opublikowanych w internecie. W zakładce Transakcje 
zgromadzone są także wszystkie płatności zarejestrowane na udostępnionych 
aplikacji kontach, z opinii długotrwałych użytkowników wynika że jakość 
automatycznej kategoryzacji jest słaba \cite{porownanieaplikacji1}
\cite{porownanieaplikacji2}\cite{porownanieaplikacji3}
\cite{porownanieaplikacji4}\cite{porownanieaplikacji5}. Użytkownik może je 
oznaczać etykietami (tag), dodawać do nich notatki czy wykluczać z zestawień. 
W zakładce Miesiąc aplikacja generuje podsumowanie przychodów i wydatków, 
pozwala utworzyć tygodniowe cele wydatków oraz budżet (a nawet kilka 
jednocześnie) przy użyciu prostego kreatora. Kreator budżetu wydaje się bardzo 
prosty w obsłudze, samodzielnie szacuje przychód na podstawie dostępnych danych 
który następnie użytkownik zatwierdza lub nadpisuje dowolną wartością, w drugim 
kroku użytkownik definiuje samodzielnie listę kategorii oraz podkategorii 
wydatków co niestety wymaga od wcześniejszego przygotowania. Użytkownik może 
dodać cel finansowy podając nazwę, wartość, oraz datę kiedy powinien być 
spełniony, następnie wybiera konto bankowe które chce z nim powiązać - jego 
spełnienie śledzi porównując ilość środków na koncie z wyznaczonym celem. 
Dodatkową opcją są przypomnienia o opłacie nadchodzących rachunków w postaci 
powiadomień push, wiadomości e-mail lub wydarzeń w kalendarzu.}

\begin{figure}[H]           %requires float package
    \centering
    \includegraphics[width=12cm]{figures/pcmag_mintmobile_05OMSsUmroXJ6F6sETKpH9R-50.fit_lim.size_1152x.jpg}
    \caption{https://www.pcmag.com/reviews/mintcom Mint w wersji mobilnej}
    \label{mintmobile}
\end{figure}

{Wersja w przeglądarce pozwala dodać nieruchomości i pojazdy zaliczane do 
całkowitej wartości netto, porady poprawy zdolności kredytowej i wartość 
inwestycji na dodanych rachunkach inwestycyjnych. Zakładka Trends zawiera 
kilka predefiniowanych wykresów wizualizujących dane, zawierających dość ogólne 
dane.}

\begin{figure}[H]           %requires float package
    \centering
    \includegraphics[width=12cm]{figures/pcmag_mintweb_05OMSsUmroXJ6F6sETKpH9R-51.fit_lim.size_1152x.jpg}
    \caption{https://www.pcmag.com/reviews/mintcom Mint w wersji na przeglądarki}
    \label{fig:mintweb}
\end{figure}

{Mint integruje także dodatkowe usługi jak wyliczanie zdolności kredytowej przez 
usługę firmy TransUnion, czy negocjację umów przez usługę firmy BillShark. 
Aplikacja sama w sobie nie zawiera zintegrowanego samouczka bądź instrukcji, 
informacje wymagane zdaniem twórców skrótowo prezentowane są kontekstowo w miejscu w 
którym są wymagane.}

{Podsumowując aplikacja Mint jest prostota w obsłudze, wymaga minimum dodatkowej
 uwagi od użytkownika, jest bardzo przejrzysta, estetyczna i nowoczesna. Jednak 
wstępna konfiguracja wymaga od użytkownika nadania jej dostępu do kont które 
zawierają wiele wrażliwych danych co jest problemem z uwagi na prywatność, 
ponadto użytkownik musi samodzielnie z góry określić wydatki na poszczególne 
kategorie zanim aplikacja udostępni mu wartościowe informacje o budżecie.}

%Trimmed content, leaving it here in case 3 examples are not enough 
%\section{\customstylesection{Aplikacje dedykowane - Goodbudget}}
%Worknotes
%{mobilna i desktopowa, darmowa (historia rok wstecz) i płatna 
%(rozszerzona, historia 7 lat wstecz), system kopertowy (dzielenie przychodów na 
%koszyki) z priorytetyzacją i monitoringiem, dobra do współdzielonego budżetu 
%(synchronizacja między kontami wielu użytkowników), wskazuje źródła edukacyjne, 
%dobre wsparcie i często rozszerzana, ręczne dodawanie wydatków, automatyczna i 
%ręczna kategoryzacja, [ręczny?] import danych,}

% 3.4. Koncepcja własnego rozwiązania (2-5 stron)
% http://siminskionline.pl/seminarium-inzynierskie/struktura-pracy-inzynierskiej/koncepcja-wlasnego-rozwiazania/
% Migrate and modernise previous description https://github.com/MarcinNowak94/Righten/commit/d8c84ff1961b77599568fe220021e3e78387e193 
\chapter{\customstylechapter{Koncepcja własnego rozwiązania}}
{Rozdział ten opisuje koncepcję rozwiązania problemów opisanych we wstępie na 
podstawie wniosków wyciągniętych z analizy istniejących rozwiązań tak, by 
wkorzystywało możliwie jak najwięcej zalet, adresowało jak najwięcej uwag i 
uzupełniało braki funckji w obecnie dostępnych na ruynku aplikacjach.}

%m.in: 
% mile widziane schematy ideowe, graficzne ilustracje przedtswionych koncepcji
% można wskazać zagadnienia możliwe do wykonania, choć wykraczające poza ramy prac
% wskazać i uzasadnić wybór metod i narzędzi pracy
% koncepcja licencjonowania, dystrybucji, jacy odbiorcy
\section{\customstylesection{Koncepcja rozwiązania użytkowego}}
{Rozwiązaniem powinna być aplikacja przeglądarkowa, co pozwoli trafić do 
szerszej grupy użytkowników minimalizując próg wejścia. Projekt aplikacji 
zakłada interakcję z wieloma użytkownikami jednocześnie. Aplikacja przechowywać 
będzie dane finansowe które są danymi wrażliwymi. Dostęp do danych będzie 
wymagany w krótkich okresach zapisu danych z pamięci podręcznej aplikacji do 
bazy oraz odpytania bazy o dane. Aplikacja udostępniać będzie użytkownikom 
interfejs do wprowadzania, edycji i usuwania danych jak przychody, rachunki 
(płatności stałe), wydatki oraz kategorie po których będą grupowane - typy 
produktów i produkty. Użytkownikom udostępnione zostaną predefiniowane raporty 
złożone z wizualizacji, statystyk i danych analitycznych.}

\medskip
{Istnieje też garstka mile widzianych funkcji które są obecnie poza zakresem 
projektu, są to między innymi: Możliwość oceny funkcji przez 
użytkownika oraz panel zgłoszeń propozycji i problemów, co pozwoli na 
ukierunkowanie rozwoju aplikacji w stronę najbardziej przydatnych i potrzebnych 
w danym momencie rozwiązań. W dalszym etapie rozwoju aplikacji można także 
wdrożyć moduł predykcji przyszłych wydatków w oparciu o dane historyczne. 
Kolejnym obszarem z potencjałem rozwoju jest wprowadzanie danych - aby ułatwić 
użytkowanie aplikacji można utworzyć moduł importu danych z plików w popularnym 
standardowym formacie jak CSV \cite{CSV}, a w dalszej perspektywie nawet funkcje
 ekstrakcji danych z obrazów co pozwala na wprowadzanie danych bezpośrednio z 
faktur, zdjęć, rachunków, paragonów, pasków wynagrodzenia. Możliwość określenia 
własnych progów wydatków które będą uwzględniane na wizualizacjach, oraz celi 
finansowych których postęp realizacji będzie mógł śledzić.}

{W fazie rozwoju na potrzeby pracy inżynierskiej projekt będzie udostępniony 
publicznie w wersji open source, dlatego wstępnie interfejs powinien być w 
języku angielskim aby poszerzyć grono potencjalnych użytkowników, poprawić 
czytelność projektu i ułatwić współpracę podczas rozwijania kodu w dalszych 
etapach. Nie jest to natomiast docelowa jedyna wersja językowa - implementację 
wyboru wersji językowej, tłumaczenie interfejsu na kilka popularnych języków 
(manualnie lub maszynowo) pozostawiono jako funkcję dodatkową, opcjonalną.}

% postać aplikacji - www/mobilna/desktop/hybrydowa?
% ogólny szkic architektury
% krótko narzędzia realizacji
% opcjonalne: dyskusja możliwych metod, technik i narzędzi realizacji pracy
% Nie wchodzić głęboko w zagadnienia technologiczne
\section{\customstylesection{Koncepcja rozwiązania technologicznego}}
{Projekt zostanie zrealizowany w podejściu LEAN \cite{LEAN} i metodologii prac 
Kanban \cite{Kanban} z wykorzystaniem usługi Trello \cite{Trello} widocznej na 
rysunku \ref*{fig:trello}. Jako że wybrano model przyrostowy 
\cite{Model Przyrostowy} priorytet zadań określać będzie klasyfikacja MoSCoW 
\cite{MOSCOW} poglądowo przedstawiony na rysunku wspierana podejściem opartym 
o matrycę Eisenhowera\cite{MatrycaEisenhowera} co pozwoli dostarczyć tak zwany 
Minimalny Wystarczający Produkt (MVP, Minimal Viable Product) \cite{MVP}. 
Zgodnie z dobrymi praktykami panującymi w inżynierii informatycznej w trakcie 
rozwoju aplikacji wykorzystany zostanie system kontroli źródła git \cite{GIT}, 
a kod projektu przechowywany będzie na portalu GitHub \cite{GitHub} (w 
repozytorium Righten \cite{GITRighten} - z poprzednimi wersjami projektu można 
się zapoznać w repozytorium DatabaseShenanigans \cite{GITBudgeterApp} a jego 
dokumentacją w Budgeter \cite{GITBudgeterDoc}). W projektowanym rozwiązaniu 
preferowane będą technologie i rozwiązania darmowe oraz open source.}

\begin{figure}[H]           %requires float package
    \centering
    \includegraphics[width=12cm]{figures/trello_kanban.png}
    \caption{Pirorytetyzacja MoSCoW}
    \label{fig:trello}
\end{figure}

\begin{figure}[H]           %requires float package
    \centering
    \includegraphics[width=12cm]{figures/MoSCoW-01.png}
    \caption{Pirorytetyzacja MoSCoW}
    \label{fig:moscow}
\end{figure}

\medskip
{Proponowane rozwiązanie powinno przyjąć formę trójwarstwowej aplikacji 
przeglądarkowej, rysunek \ref{fig:architeturatrojwarstwowa} przedstawia 
poglądową architekturę. Ponieważ z aplikacji ma korzystać wielu użytkowników 
jednocześnie wymagana jest architektura multitenant \cite{multitenant}, która 
pozwala korzystać wielu użytkownikom z tej samej bazy aplikacji - proponowana 
technologia to PostgreSQL \cite{PostgreSQL}. Z uwagi na prywatność danych każdy 
z użytkowników docelowo będzie korzystał z własnego schematu w bazie danych 
aplikacji przechowywanej na serwerze co przedstawia rysunek 
\ref{fig:multitenant}, jest to jednak rozwiązanie które trudno wdrożyć, dlatego 
w fazie projektowej która jest przedmiotem tej pracy zastosowano uproszczenie w 
postaci pojedynczego domyślnego schematu danych dla każdego użytkownika 
(słowem: wszyscy użytkownicy mają dostęp do tych samych danych).}

\begin{figure}[H]           %requires float package
    \centering
    \includegraphics[width=12cm]{figures/framwork-gigr-pl_architektura_www.jpg}
    \caption{http://framework.gigr.pl/ Architektura Trójwarstwowa}
    \label{fig:architeturatrojwarstwowa}
\end{figure}

\begin{figure}[H]           %requires float package
    \centering
    \includegraphics[width=12cm]{figures/multitenant_4.png}
    \caption{Microsoft, Architektura współdzielona baza, rozdzielne schematy}
    \label{fig:multitenant}
\end{figure}


% 3.5. Projekt ogólny (10-25 stron.)
% http://siminskionline.pl/seminarium-inzynierskie/struktura-pracy-inzynierskiej/projekt-ogolny/
\chapter{\customstylechapter{Projekt ogólny}}
{Rozdział ten opisuje ogólną koncepcję organizacji systemu, plan jego 
architektury, przechowywanie danych, ogólny plan interfejsu użytkownika oraz 
metody i narzędzia realizacji.}

%opis funkcji udostępnianych przez system
% - mile widziany diagram przypadkó użycia - Funkcje systemu odpowiadają 
%   zazwyczaj przypadkom użycia
% - można zaprezentować wymagania jako numerowaną listę, tabelę
% - wymagania niefunkcjonalne - aspekty systemu które nie przekładają się 
%   bezpośrednio na akcje wykonywane przez system (architektura, bezpieczeństwo,
%   ergonomia, kolorystyka itd.)

\section{\customstylesection{Specyfikacja wymagań funkcjonalnych i niefunkcjonalnych}}
{Zestawienie funkcji które powinien spełniać program, wraz z informacją które 
z nich zostały spełnione. Nagłówki z powodu objętości zostały skrócone, legenda:}

{PRIO - Priorytet w jednej z kategorii MOSCOW \cite{MOSCOW}}

{IMPL - Oznaczenie czy funkcję wdrożono}

%#TODO: try tabularray as suggested
% Wrapping as per: https://stackoverflow.com/questions/790932/how-to-wrap-text-in-latex-tables
\begin{table}[H] %https://www.overleaf.com/learn/latex/Positioning_images_and_tables
    \caption{Wymagania niefunkcjonalne}
    \label{Wymagania niefunkcjonalne}
    \footnotesize
    \begin{tabular}{|p{0.2\linewidth}|p{0.07\linewidth}|p{0.07\linewidth}|p{0.52\linewidth}|}  % | draws verical line
    % \usepackage{booktabs} provides different line thicknesses
    % \toprule, \midrule, \bottomrule
    \hline                  % Draw horizontal line
    % & Defines the breaks in the table 
    \customstyletable{Funkcja} & \customstyletablecentered{PRIO} & \customstyletablecentered{IMPL}& \customstyletable{Opis} \\
    \hline
    {Plik konfiguracji} & {M} & {TAK} & {Osobny plik konfiguracyjny}\\
    \hline
    {Rejestr zdarzeń} & {S} & {NIE} & {Logi z działania aplikacji}\\
    \hline
    {Instalator} & {W} & {NIE} & {Prosty instalator aplikacji}\\
    \hline
    {Aktualizacje} & {W} & {NIE} & {Automatyczne sprawdzanie wersji i aktualizacja}\\
    \hline
    \end{tabular}
\end{table}

\begin{table}[H] %https://www.overleaf.com/learn/latex/Positioning_images_and_tables
    \caption{Wymagania funkcjonalne}
    \label{Wymagania funkcjonalne}
    \footnotesize
    \begin{tabular}{|p{0.2\linewidth}|p{0.07\linewidth}|p{0.07\linewidth}|p{0.52\linewidth}|}  % | draws verical line
    % \usepackage{booktabs} provides different line thicknesses
    % \toprule, \midrule, \bottomrule
    \hline                  % Draw horizontal line
    % & Defines the breaks in the table 
    \customstyletable{Funkcja} & \customstyletablecentered{PRIO} & \customstyletablecentered{IMPL}& \customstyletable{Opis} \\
    \hline
    {Dodawanie danych} & {M} & {TAK} & {Dodawanie danych}\\
    \hline
    {Podsumowanie wydatków} & {M} & {TAK} & {Okresowe podsumowanie wydatków}\\
    \hline
    {Podsumowanie przychodów} & {M} & {TAK} & {Okresowe podsumowanie przychodów}\\
    \hline
    {Statystyki typów} & {C} & {TAK} & {Statystyki wydatków na dany typ produktu}\\
    \hline
    {Statystyki produktów} & {C} & {TAK} & {Statystyki wydatków na dany produkt}\\
    \hline
    {Bilans okresowy} & {M} & {TAK} & {Okresowy bilans zysków i strat}\\
    \hline
    {Definiowanie produktów} & {M} & {TAK} & {Definiowanie produktów}\\
    \hline
    {Definiowanie przychodów} & {M} & {TAK} & {Definiowanie przychodów}\\
    \hline
    {Definiowanie typów produktów} & {M} & {TAK} & {Definiowanie typów produktów}\\
    \hline
    {Definiowanie typów przychodów} & {C} & {NIE} & {Definiowanie typów przychodów}\\
    \hline
    {Panel konfiguracyjny} & {S} & {TAK} & {Osobny panel konfiguracyjny}\\
    \hline
    {Dostęp zdalny} & {M} & {TAK} & {Dostęp do zdalnych baz danych}\\
    \hline
    {Import danych} & {S} & {NIE} & {Import danych w standardowym formacie}\\
    \hline
    {Walidacja danych} & {M} & {TAK} & {Potwierdzenie jakości danych}\\
    \hline
    {Eksport danych} & {C} & {NIE} & {Eksport danych do standardowego formatu}\\
    \hline
    {Trendy} & {W} & {NIE} & {Predykcja trendów wydatków i wpływów}\\
    \hline
    {Porady} & {C} & {TAK} & {Porady dla użytkownika dotyczące usprawnień budżetu}\\
    \hline
    {Wiele użytkowników} & {M} & {TAK} & {Wsparcie dla wielu użytkowników jednocześnie}\\
    \hline
    {Lokalizacje} & {C} & {NIE} & {Wersje językowe interfejsu do wyboru}\\
    \hline
    {Personalizacja interfejsu} & {W} & {NIE} & {Personalizacja interfejsu użytkownika}\\
    \hline
    \end{tabular}
\end{table}

%(system klasy desktop, mobilny, internetowy)
%rysunki ideowe i poglądowe
%Ogólny opis przeznaczenia i roli poszczególnych elementów i komunikacja między nimi
\section{\customstylesection{Architektura systemu}}
{Przedmiotem projektu będzie system klasy internetowej - aplikacja internetowa 
dostępna w przeglądarce w architekturze trójwarstwowej. Aplikację będzie można 
umieścić na dedykowanym serwerze lub jeśli dodatkowy cel konteneryzacji zostanie
 zrealizowany - na dowolnej maszynie na której udostępniony będzie kod źródłowy 
oraz zainstalowany Docker. Aplikacja podzielona będzie na interfejs użytkownika 
(frontend) odpowiadający za interakcję z użytkownikiem i walidację danych, kod 
na serwerze (backend) zawierający logikę działania aplikacji i komunikację z 
warstwą dostępu do danych którą będzie baza danych z danymi aplikacji i 
użytkowników. Wszystkie komponenty aplikacji docelowo działać będą na 
pojedynczej maszynie, jednak nic nie stoi na przeszkodzie aby w przyszłości 
jeżeli zajdzie taka potrzeba wydzielić poszczególne komponenty na osobnych 
maszynach.}

{Aby uprościć logikę aplikacji i zwiększyć jej wydajność ciężar przetwarzania 
danych zostanie przerzucony na warstwę bazy danych. Zadanie to przejmą widoki 
napisane w języku SQL\cite{SQL} które odpytane przez aplikacje przetwarzają 
aplikacje w locie w bardzo wydajny sposób.}

%Przegląd możliwych metod i narzędzi realizacji
%Wybrać jedno rozwiązanie i/lub uzasadnić wybór rozwiązania
%Krótka charakterystyka narzędzi i określenie wersji
%krótko i konkretnie, szczegóły tylko jeśli są istotne dla projektu
\section{\customstylesection{Metody i narzędzia realizacji}}
{Do implementacji kodu aplikacji użyty zostanie język Python \cite{Python}, 
głównie przez wzgląd na walory edukacyjne i prostej, ekspresywnej składni co 
zwiększy czytelność kodu i zmniejszy poziom złożoności aplikacji. Funkcje }

{Warstwa interfejsu użytkownika (frontend) składać się będzie z szablonów HTML z
 wykorzystaniem silnika szablonów jinja2 \cite{jinja}, wzbogacone o funkcje 
udostępniane przez framework Bootstrap \cite{Bootstrap} który umożliwi 
uzupełnianie danych w interfejsie użytkownika funkcjami napisanymi w 
JavaScript \cite{JavaScript}. Do obsługi formularzy wprowadzania danych posłuży 
framework WTForms \cite{WTForms}, a wizualizacja danych będzie obsługiwana 
dzięki bibliotece chart.js \cite{chart.js}.}

{Do implementacji funkcji serwera sieciowego (backend) posłuży framework
 Flask \cite{Flask} wraz z dodatkowymi wtyczkami do obsługi poszczególnych 
funkcji (m.in.: Flask-Login \cite{Flask-Login}) oraz określonymi w trakcie 
pisania aplikacji wymaganymi bibliotekami.}

{Wstępna warstwa bazy danych przygotowana na lokalnej instancji 
SQLite3 \cite{SQLite}, w dalszej części projektu zostanie zmigrowana do 
docelowej technologii jaką jest PostgreSQL \cite{PostgreSQL} - rozwiązanie takie
 przyjęto ponieważ instancja PostgreSQL do działania wymaga serwera oraz osobnej
 aplikacji pgadmin \cite{pgAdmin} do zarządzania nią.}

{Domyślnie rozwiązanie będzie działało natywnie na serwerze lub maszynie 
wirtualnej, jednak opcjonalnie w ramach dalszego rozwoju przewidziano aby 
środowisko aplikacji powoływać dynamicznie z wykorzystaniem platformy 
uruchomieniowej Docker \cite{Docker}.}

{Do tworzenia dokumentacji wykorzystany zostanie pakiet narzędzi open source, 
między innymi będą to: StarUML \cite{StarUML} do tworzenia diagramów, pgAdmin 
\cite{pgAdmin} do tworzenia diagramów na podstawie encji bazy danych. 
Dokumentacja zostanie spisana w języku LaTex \cite{LaTeX} wraz z gamą 
oficjalnych rozszerzeń dostępnych w sieci - zarówno kod jak i dokumentacja 
spisana w środowisku VSCode \cite{VSCode}.}

%jak będziemy zapisywać dane trwale
%model bazy danych i jego szczegóły
%dla baz relacyjnych - konceptualny, logiczny i fizyczny, jeśli dużo się nie różnią ten najbardziej pełny
%ERD (Entity-Relationship Diagram) i opis przenzaczenia tabel, klucze główne i obce
\section{\customstylesection{Koncepcja przechowywania danych}}
{W warstwie przechowywania danych aplikacji do trwałego zapisu posłuży baza 
danych w technologii PostgreSQL \cite{PostgreSQL}. Podstawowy model schematu 
bazy danych przedstawia rysunek \ref{fig:rightenerdtabele}, który prezentuje 
Diagram Związków Encji (ERD, Entity-Relationship Diagram) tabel w szablonowym 
schemacie bazy danych aplikacji. Widoki prezentuje rysunek 
\ref{fig:rightenerdwidoki}, przetwarzają one dane które prezentowane są 
użytkownikowi w aplikacji.}

\begin{figure}[H]           %requires float package
    \centering
    \includegraphics[width=12cm]{figures/RightenDB_EntityRelationDiagram.png}
    \caption{Diagram Związków Encji (ERD, Entity-Relationship Diagram)}
    \label{fig:rightenerdtabele}
\end{figure}

\begin{figure}[H]           %requires float package
    \centering
    \includegraphics[width=12cm]{figures/Righten_Finances-db_Views.png}
    \caption{Widoki w bazie danych aplikacji}
    \label{fig:rightenerdwidoki}
\end{figure}

{Podstawowe tabele aplikacji: Tabela Income to zbiór przychodów, natomiast 
tabela Bills to zbiór okresowych wydatków stałych. Tabela ProductTypes zawiera 
dane o typach produktów, Tabela Products zawiera dane o Produktach które 
powiązane są z typami. Tabela Expenditures przechowuje zbiór wydatków 
okazjonalnych na określone produkty.}

{Tabele dotyczące użytkownika: W tabeli Users przechowywane są dane o 
użytkownikach aplikacji jak UUID, nazwa, hash hasła i ustawienia 
administracyjne, ich identyfikatorem jest UUID. Tabela UserSettings zawiera 
ustawienia prywatne użytkownika - jej struktura jest rozwijana wraz z rozwojem 
funkcji projektu.}

{Tabele pomocnicze i tymczasowe: są to tabele robocze do zarządzania danymi w 
trakcie rozwoju, ich ilość, nazwy, struktura oraz przeznaczenie mogą się 
zmieniać. Kiedy projektowana funkcja zostanie ukończona tabela wchodzi do użytku
 w ramach którejś z wcześniej opisanych kategorii.}

%Organizacja interfejsu użytkownika - rysunki z komentarzem
%szkice(mockup) i projekty, nie gotowe zrzuty ekranu
%dla aplikacji responsywnych przykłady na różnych urządzeniach 
\section{\customstylesection{Projekt interfejsu użytkownika}}
{Interfejs użytkownika będzie prosty, minimalistyczny aby ułatwić użytkownikowi 
poruszanie się po aplikacji i zmniejszyć obciążenie poznawcze. Dzięki temu opcje
 powinny być łatwo dostępne, a ryzyko że użytkownik nie znajdzie opcji której 
poszukuje minimalne.}

{Pierwszym ekranem który napotka użytkownik jest tak zwany splashscreen którego 
projekt prezentuje rysunek \ref*{fig:uiprojectsplash}, po wejściu w opcję login 
użytkownik trafi na ekran startowy który prezentuje rysunek 
\ref*{fig:uiprojectstart}. Rysunek \ref*{fig:uiprojectlogin} prezentuje ekran 
logowania. Zalogowany użytkownik będzie mógł zmienić swoje ustawienia na ekranie
 ustawień zobrazowanym przez rysunek \ref*{fig:uiprojectsettings} - funkcja ta 
jest funkcją dodatkową, zostanie zrealizowana jeśli wszystkie funkcje 
spełniające podstawowe założenia i wymagane do prawidłowego działania aplikacji 
zostaną wdrożone. Dodatkowo w aplikacji powstaną ekranu poradnika na wzór 
prezentowanego na rysunku \ref*{fig:uiprojectguide} z podstawowymi informacjami 
o zarządzaniu finansami - we wstępnej fazie aby pokazać potencjał rozwiązania 
ekran będzie tylko jeden, pozostałe pozostaną puste lub zostaną wypełnione 
domyślnym tekstem, który w miarę rozwoju aplikacji zastąpią docelowe informacje 
ze sprawdzonych źródeł o dobrej reputacji.}

\begin{figure}[H]           %requires float package
    \centering
    \includegraphics[width=12cm]{figures/Righten_UI_sketch_splashscreen.png}
    \caption{Projekt ekranu początkowego (splashscreen)}
    \label{fig:uiprojectsplash}
\end{figure}

\begin{figure}[H]           %requires float package
    \centering
    \includegraphics[width=12cm]{figures/Righten_UI_sketch_startscreen.png}
    \caption{Projekt ekranu pierwszej interakcji}
    \label{fig:uiprojectstart}
\end{figure}

\begin{figure}[H]           %requires float package
    \centering
    \includegraphics[width=12cm]{figures/Righten_UI_sketch_loginscreen.png}
    \caption{Projekt ekranu logowania}
    \label{fig:uiprojectlogin}
\end{figure}

\begin{figure}[H]           %requires float package
    \centering
    \includegraphics[width=12cm]{figures/Righten_UI_sketch_settings.png}
    \caption{Projekt ekranu dodawania danych}
    \label{fig:uiprojectsettings}
\end{figure}

\begin{figure}[H]           %requires float package
    \centering
    \includegraphics[width=12cm]{figures/Righten_UI_sketch_guide.png}
    \caption{Projekt ekranu poradnika}
    \label{fig:uiprojectguide}
\end{figure}

{Ekran zarządzania danymi prezentuje rysunek \ref*{fig:uiprojectdatamanagement},
jest to ekran który pozwala edytować dane dodane przez ekran dodawania danych 
przedstawiony na rysunku \ref*{fig:uiprojectadddata}. W obu ekranach dane 
wprowadzane do aplikacji będą walidowane po stronie klienta przez logikę zaszytą
 w formularzach do wprowadzania danych dostarczoną przez wybrane rozwiązanie.}

\begin{figure}[H]           %requires float package
    \centering
    \includegraphics[width=12cm]{figures/Righten_UI_sketch_datamanagement.png}
    \caption{Projekt ekranu zarządzania danymi}
    \label{fig:uiprojectdatamanagement}
\end{figure}

\begin{figure}[H]           %requires float package
    \centering
    \includegraphics[width=12cm]{figures/Righten_UI_sketch_adddata.png}
    \caption{Projekt ekranu dodawania danych}
    \label{fig:uiprojectadddata}
\end{figure}

{Na pozostałe ekrany aplikacji składać się będą różnego typu wizualizacje, 
poglądowo prezentowane przez projekt na rysunku 
\ref*{fig:uiprojectvisualizations}. Liczba i złożoność wizualizacji zależeć 
będzie od nakładów pracy wymaganych do ich wdrożenia. Na wstępie użytkownikom 
udostępnione zostaną wyłącznie podstawowe wizualizacje, natomiast bardziej 
złożone i nowe uzupełniane będą z czasem w trakcie rozwoju aplikacji, w oparciu 
o uwagi użytkowników. Takie podejście zagwarantuje że aplikacja będzie 
maksymalnie przydatna do celów do których została stworzona - pomocy 
użytkownikom przy zarządzaniu budżetem.}

\begin{figure}[H]           %requires float package
    \centering
    \includegraphics[width=12cm]{figures/Righten_UI_sketch_viusalizations.png}
    \caption{Projekt ekranu dodawania danych}
    \label{fig:uiprojectvisualizations}
\end{figure}

%#TODO:
%3.6. Dokumentacja techniczna
% http://siminskionline.pl/seminarium-inzynierskie/struktura-pracy-inzynierskiej/projekt-techniczny/
%struktura zależna od wybranych rozwiązań, metod i architektury
%diagramy projektowe - np.: diagram hierarchii klas, diagramy sekwencji, kolaboracji
%opisy metod i stosowane algorytmy
\chapter{\customstylechapter{Dokumentacja techniczna}}
{Znaczniki czasu przechowywane będą w formacie czasu zgodnym ze standardem 
ISO 8601 \cite{ISO 8601} - jednolity format ułatwi obsługę i przetwarzanie 
danych.}

{Dane między warstwami backend i frontend muszą być przesyłane w standardowym 
formacie aby zapewnić współpracę z dostępnymi na rynku bibliotekami i uprościć 
implementacje funkcji które będą je obsługiwać. Na potrzeby aplikacji wybrano 
format JSON \cite{JSON}.}

%zaprezentować: pakiety, moduły, biblioteki, klasy
%składowe omówić, jej: nazwa, rola w projekcie, przeznaczenie, struktura wewnętrzna, opis
%przedstawić wybrane (istotne, ciekawe, nietypowe) fragmenty kodu, pomijać trywialne
%listingi wstawić jako sformatowany tekst
%nie zamieszczać długich listingów - kilka linijek, opis, kilka linijek, opis

\section{\customstylesection{Schemat rzeczywistej struktury systemu}}
{}

\section{\customstylesection{Wybrane fragmenty kodu aplikacji}}
{}

\section{\customstylesection{Wybrane fragmenty kodu obiektów bazy danych}}
{}

%3.7. Testy i weryfikacja systemu (2-5str)
% http://siminskionline.pl/seminarium-inzynierskie/struktura-pracy-inzynierskiej/testy/
%brak tego rozdziału jest istotnym mankamentem pracy co obniży jej ocenę
%Jakie testy zaplanowano i przygotowano, jak je zrealizowano i jakie były rezultaty
%Jednostkowe, integracyjne, akceptacyjne i responsywności - jeśli TDD tylko wybrane testy
%Tylko wybrane testy, zwłąszcza tam gdzie pojawiły się błędy, ich przyczyna i usunięcie
%forma: tekst i tabelki przypadków testowych
\chapter{\customstylechapter{Testy i weryfikacja systemu}}
{}

%#TODO:opisać problem z solą przy szyfrowaniu haseł bcrypt
\section{\customstylesection{Najciekawsze wykryte błędy}}
{}

%3.8. Przykładowy scenariusz wykorzystania systemu (4-10str)
% http://siminskionline.pl/seminarium-inzynierskie/struktura-pracy-inzynierskiej/scenariusz/
%efekty pracy, skoncentrować się na użytkowych aspektach rozwiązania
%przedstawienie typowego wykorzystania systemu, krok po kroku z czytelnymi grafikami
%wybrać elementy najciekawsze, nietypowe, ważne
%zrealizować przypadki użycia Jeśli opisano w Koncepcji własnego rozwiązania lub opisie wymagań funkcjonalnych
%poglądowo, nie drobiazgowo ani nie samymi grafikami. rysunek/dwa +5-6 zdań
\chapter{\customstylechapter{Przykładowy scenariusz wykorzystania systemu}}
{}

%3.9. Zakończenie (1-2str, najlepiej 1,5str)
% http://siminskionline.pl/seminarium-inzynierskie/struktura-pracy-inzynierskiej/zakonczenie/
%Podsumowanie wyników pracy, potwierdzenie realizacji celu
%Wady i zalety rozwiązania i możliwości jego rozwoju - nie dzielić na podpunkty
%Komplementarne do wstępu - zarówno wstęp jak i zakończenie będą uważnie czytane
% przez recenzenta 
% -"Celem pracy było..."
% -0,3-0,5str Opis właściwości funkcjonalnych zaproponowanego rozwiązania
% -0,3-0,5str Przypomnienie technologii, narzędzi, bibliotek
% - Negatywne i pozytywne spostrzeżenia z realizacji pracy - ciekawe, zaskakujące, trudności.
%   Jak rozwiązanie sprawdza się w praktyce (wdrożone - opinie użytkowników, niewdrożone - czy i kiedy)
% pochwalić co dobrze wyszło, nie ukrywać wad - wymienić je i opisać jak je poprawić.
% - Opis kierunkó rozwoju funkcjonalnego i technologicznego
\chapter{\customstylechapter{Zakończenie}}
{Celem pracy było zaprojektowanie i realizacja modułu analitycznego aplikacji 
ułatwiającej jej użytkownikom zarządzanie budżetem domowym dostarczając 
narzędzia do analizy wpływów i wydatków, wizualizacji trendów oraz automatycznie 
kategoryzującej wprowadzone dane.}

% [#TODO]:
% 3.13.Ewentualne załączniki


% #TODO: Remove unused sources
% 3.10.Bibliografia
\begin{thebibliography} {books}
    %Books
    \bibitem{najbogatszyczlowiekwbabilonie} George S. Clason, (2021) Najbogatszy człowiek w Babilonie, ISBN: 978-83-67060-04-2
    \bibitem{finansowaforteca} Marcin Iwuć, (2020) Finansowa Forteca, ISBN: 978-83-958468-0-9
    \bibitem{budzetdomowypodkontrola} Krzysztof Piotr Łabenda, (2011) Budżet domowy pod kontrolą. Jak rozsądnie wydawać, oszczędzać i inwestować pieniądze, ISBN: 978-83-246-3626-6

    %Network resources
    \bibitem{pythonautomate} Al Sweigart, Automate the Boring Stuff with Python \raggedright\url{
        https://automatetheboringstuff.com/#toc}
    \bibitem{wiki_ekonomia} Wikipedia, Nauki Ekonomiczne \raggedright\url{
        https://pl.wikipedia.org/wiki/Nauki_ekonomiczne}
    \bibitem{zapaśćekonomiczna} Wikipedia, Economic collapse \raggedright\url{
        https://en.wikipedia.org/wiki/Economic_collapse}
    \bibitem{gussytuacjabudzetowa} Główny Urząd Statystyczny, Budżety gospodarstw domowych w 2022 roku \raggedright\url{
        https://stat.gov.pl/obszary-tematyczne/warunki-zycia/dochody-wydatki-i-warunki-zycia-ludnosci/budzety-gospodarstw-domowych-w-2022-roku,9,21.html}
    \bibitem{edukacjafinansowawszkołach} Łukasz Grygiel, Jak wygląda edukacja finansowa dzieci i młodzieży w Polsce? \raggedright\url{
        https://web.archive.org/web/20230529115945/https://lukaszgrygiel.com/edukacja-finansowa-dzieci-i-mlodziezy/}
    \bibitem{edukacjafinansowamlodziezy} Bank Pekao, Raport Banku Pekao: „Dziecięcy świat finansów - jak rynek finansowy odpowiada na potrzeby najmłodszych klientów”. \raggedright\url{
        https://www.pekao.com.pl/o-banku/aktualnosci/d4e423aa-0ba4-4bde-8a0a-7ff3a17a9793/raport-banku-pekao-dzieciecy-swiat-finansow-jak-rynek-finansowy-odpowiada-na-potrzeby-najmodszych-klientow.html}
    \bibitem{portfelpolakawpandemii} Krajowy Rejestr Długów, Portfel statystycznego Polaka w pandemii \raggedright\url{
        https://krd.pl/centrum-prasowe/raporty/2022/portfel-statystycznego-polaka-w-pandemii}
    \bibitem{koszty2010-22} Warsaw Enterprise Institute, [RAPORT] Żegnajcie niskie ceny? Koszty życia i poziom cen w Polsce na tle krajów UE w latach 2010–2022 \raggedright\url{
        https://wei.org.pl/2022/aktualnosci/wiktorwojciechowski/raport-zegnajcie-niskie-ceny-koszty-zycia-i-poziom-cen-w-polsce-na-tle-krajow-ue-w-latach-2010-2022/}
    \bibitem{weiinflacja} Warsaw Enterprise Institute, [RAPORT] Jak inflacja zubaża Polaków? \raggedright\url{
        https://wei.org.pl/2023/publikacje/raporty/mateusz-benedyk/raport-jak-inflacja-zubaza-polakow/}
    \bibitem{o24_budzetowanie} Opcje24, Budzetowanie \raggedright\url{
        https://www.opcje24h.pl/budzetowanie-przewodnik-planowanie-budzetu/}
    \bibitem{iwućbudżet} /marciniwuc.com, Budżet domowy krok po kroku \raggedright\url{
        https://marciniwuc.com/budzet-domowy-krok-po-kroku/}
    \bibitem{ingbudżet} ING Bank Śląski, Jak zapanować nad budżetem domowym? \raggedright\url{
        https://spolecznosc.ing.pl/-/Blog/Jak-zapanowa\%C4\%87-nad-bud\%C5\%BCetem-domowym/ba-p/3968}
    \bibitem{budget} The Balance, Understanding Budgeting \& Personal Finance\raggedright\url{
        https://www.thebalancemoney.com/personal-finance-budget-4802696}
    \bibitem{mintbudget} www.mint.intuit.com, Budgeting 101 \raggedright\url{
        https://mint.intuit.com/blog/category/budgeting/}
    \bibitem{pozytywnewzmocnienie} Simply Scholar, Ltd., Positive Reinforcement: What Is It and How Does It Work? \raggedright\url{
        https://www.simplypsychology.org/positive-reinforcement.html}
    \bibitem{multitenant} Frederick Chong, Gianpaolo Carraro, and Roger Wolter, Microsoft Corporation, Multi‐Tenant Data Architecture \raggedright\url{
        https://ramblingsofraju.com/wp-content/uploads/2016/08/Multi-Tenant-Data-Architecture.pdf}

    %Evaluated solutions
    \bibitem{santandercentrum24} Santander, Centrum24.pl \raggedright\url{
        https://www.centrum24.pl/}
    \bibitem{mint} Inuit inc., Mint \raggedright\url{
        https://mint.intuit.com/}
    \bibitem{mintrecenzja} Ryan McGregor, Mint Budgeting App: How to Setup and Use a Budget (BEST WAY) \raggedright\url{
        https://www.youtube.com/watch?v=rQ_5v3BUBqQ}
    \bibitem{mintwycofanie} theverge.com, Mint is shutting down, and it’s pushing users toward Credit Karma \raggedright\url{
        https://mint.intuit.com/blog/mint-app-news/intuit-credit-karma-welcomes-minters/}
    \bibitem{goodbudget} Dayspring Partners, Goodbudget \raggedright\url{
        https://goodbudget.com/}
    \bibitem{porownanieaplikacji1} https://www.cnbc.com/select/best-free-budgeting-tools/ \raggedright\url{
        https://www.cnbc.com/select/best-free-budgeting-tools}
    \bibitem{porownanieaplikacji2} https://www.forbes.com/advisor/banking/best-budgeting-apps/ \raggedright\url{
        https://www.forbes.com/advisor/banking/best-budgeting-apps/}
    \bibitem{porownanieaplikacji3} https://www.tomsguide.com/best-picks/best-budgeting-apps \raggedright\url{
        https://www.tomsguide.com/best-picks/best-budgeting-apps}
    \bibitem{porownanieaplikacji4} https://www.investopedia.com/best-budgeting-apps-5085405 \raggedright\url{
        https://www.investopedia.com/best-budgeting-apps-5085405}
    \bibitem{porownanieaplikacji5} https://www.nerdwallet.com/article/finance/best-budget-apps \raggedright\url{
        https://www.nerdwallet.com/article/finance/best-budget-apps}
    \bibitem{tiller} Tiller Inc., Tiller \raggedright\url{
        https://www.tillerhq.com/}

    %Metodologies
    \bibitem{MOSCOW} Product Plan, MOSCOW Prioritetization \raggedright\url{
        https://www.productplan.com/glossary/moscow-prioritization/}
    \bibitem{MatrycaEisenhowera}Praca.pl, Matryca Eisenhowera - czym jest, zasada, prioryteryzacja zadań \raggedright\url{
        https://www.praca.pl/poradniki/rynek-pracy/matryca-eisenhowera-czym-jest,zasada,prioryteryzacja-zadan_pr-2012.html}
    \bibitem{MVP} Wikipedia, Minimal Viable Product \raggedright\url{
        https://en.wikipedia.org/wiki/Minimum_viable_product}
    \bibitem{Kanban} Lean Action Plan, Kanban – układ nerwowy sterowania produkcją w koncepcji Lean Manufacturing \raggedright\url{
        https://leanactionplan.pl/kanban/}
    \bibitem{LEAN} Wikipedia, Lean software development \raggedright\url{
        https://pl.wikipedia.org/wiki/Lean_software_development}
    \bibitem{Model Przyrostowy} Wikipedia, Model Przyrostowy \raggedright\url{
        https://pl.wikipedia.org/wiki/Model_przyrostowy}
    
    %Standards
    \bibitem{ISO 8601} NASA.gov, A summary of the international standard date and time notation \raggedright\url{
        https://fits.gsfc.nasa.gov/iso-time.html}
    \bibitem{CSV} Y. Shafranovich, SolidMatrix Technologies, Inc., Common Format and MIME Type for Comma-Separated Values (CSV) Files \raggedright\url{
        https://www.rfc-editor.org/rfc/rfc4180}
    \bibitem{JSON} json.org, Introducing JSON \raggedright\url{
        https://www.json.org/json-en.html}
    
    %Tools and technologies
    \bibitem{SQLite} sqlite.org, SQLite \raggedright\url{
        https://www.sqlite.org/index.html}
    \bibitem{SQL} wikipedia.org, SQL - Structured Query Language \raggedright\url{
        https://en.wikipedia.org/wiki/SQL}
    \bibitem{PostgreSQL} postgresql.org, PostgreSQL \raggedright\url{
        https://www.postgresql.org/}
    \bibitem{pgAdmin} pgadmin.org, pgAdmin \raggedright\url{
        https://www.pgadmin.org/}
    \bibitem{Docker} docker.com, Docker \raggedright\url{
        https://www.docker.com/}
    \bibitem{Python} python.org, Python \raggedright\url{
        https://www.python.org/}
    \bibitem{Trello} Atlassian, Trello.com \raggedright\url{
        https://trello.com/}
    \bibitem{StarUML} MKLabs Co.,Ltd, StarUML \raggedright\url{
        https://staruml.io/}
    \bibitem{LaTeX} The LaTeX Project \raggedright\url{
        https://www.latex-project.org/}
    %Database visualization tool used for Budgeter - used DBSchema for Righten
    %\bibitem{DataGrip} JetBrains, DataGrip \raggedright\url{
    %    https://www.jetbrains.com/datagrip/}
    \bibitem{VSCode} Microsoft, Visual Studio Code \raggedright\url{
        https://code.visualstudio.com/}
    \bibitem{GIT} git-scm.com, git \raggedright\url{
        https://git-scm.com/}
    \bibitem{GitHub} https://github.com/ \raggedright\url{
        https://github.com/}
    \bibitem{GITBudgeterApp} github.com MarcinNowak94, DatabaseShenanigans \raggedright\url{
        https://github.com/MarcinNowak94/DatabaseShenanigans}
    \bibitem{GITBudgeterDoc} github.com MarcinNowak94, budgeter \raggedright\url{
        https://github.com/MarcinNowak94/budgeter}
    \bibitem{GITRighten} github.com MarcinNowak94, Righten \raggedright\url{
        https://github.com/MarcinNowak94/Righten}
    \bibitem{JavaScript} Mozilla, JavaScript \raggedright\url{
        https://developer.mozilla.org/en-US/docs/Web/javascript}
    \bibitem{Bootstrap} getbootstrap.com, Bootstrap \raggedright\url{
        https://getbootstrap.com/}
    \bibitem{Flask} flask.palletsprojects.com, Flask \raggedright\url{
        https://flask.palletsprojects.com/en/3.0.x/}
    \bibitem{Flask-Login} https://pypi.org/project/Flask-Login/, Flask-Login \raggedright\url{
        https://pypi.org/project/Flask-Login/}
    \bibitem{WTForms} wtforms.readthedocs.io, WTForms \raggedright\url{
        https://wtforms.readthedocs.io/en/3.1.x/}
    \bibitem{jinja} jinja.palletsprojects.com, Jinja \raggedright\url{
        https://jinja.palletsprojects.com/en/3.1.x/}
    \bibitem{chart.js} www.chartjs.org, Chart.js \raggedright\url{
        https://www.chartjs.org/}
\end{thebibliography}

% 3.11.Spis rysunków
\listoffigures
% 3.12.Spis tabel
\listoftables
\lstlistoflistings

\end{large}
\end{document}

% Phase 2: Bachelors thesis ----------------------------------------------------
% [DONE]:
% 0. [SELF] Rename projects
% 1. [SELF] Change repository so code and docummentation are in one project
% 2. [SELF] Add Licence
% 3. [Initial review] Change structure to adhere to rules described here:
% http://siminskionline.pl/seminarium-inzynierskie/struktura-pracy-inzynierskiej/
% 3.1. Wstęp <-[sections moved here: Motywacja, cel i plan działania]
% 3.2. Charakterystyka/analiza problemu
% 3.3. Analiza istniejących rozwiązań
% 3.4. Koncepcja własnego rozwiązania (2-5 stron)
% 3.10.Bibliografia
% 3.11.Spis rysunków
% 3.12.Spis tabel